\section{Optimization}
\label{sec:optim}

Optimizing problem~\eqref{eq:jointstateactionprob} poses several challenges that need to be addressed.
First, we propose a relaxation of the integer constraints and the distortion function (Section~\ref{sec:relax}).
Second, we optimize this relaxation using Frank-Wolfe with a new dynamic program able to handle our tracklet constraints (Section~\ref{subsec:frank-wolfe}).
Finally, we introduce a new rounding technique to obtain an integer candidate solution to our problem (Section~\ref{sec:rounding}).

%
%
%
%

\subsection{Relaxation}
\label{sec:relax}
Problem~\eqref{eq:jointstateactionprob} is NP-hard in general~\cite{loiola07qap} due to its specific integer constraints.
%
Inspired by the approach of~\cite{Bojanowski14weakly} that was successful to approximate combinatorial optimization problems, we propose to use the tightest convex relaxation of the feasible subset of binary matrices by taking its convex hull.
As our variables now can take values in $[0,1]$, we also have to propose a consistent extension for the different cost functions to handle fractional values as input.
For the cost functions $f$ and $g$, we can directly take their expression on the relaxed set as they are already expressed as (convex) quadratic functions.
Similarly, for the joint cost function $d$ in~\eqref{eq:dist}, we use its natural bilinear relaxation:


\vspace{-3mm}
\begin{align}
d(Z_n,Y_n) = \sum_{i=1}^{M_n} \sum_{t=1}^{T_n}  \Big( &(Y_n)_{i1} Z_{nt} [t_{ni}-t]_+ \,\, + \notag \\[-3mm]
&(Y_n)_{i2} Z_{nt} [t-t_{ni}]_+ \,\, \Big),
\label{eq:relaxeddist}
\end{align}
where $t_{ni}$ denotes the video time of tracklet $i$ in clip $n$.
This relaxation is equal to the function~\eqref{eq:dist} on the integer points.
However, it is not jointly convex in~$Y$ and~$Z$, thus we have to design an appropriate optimization technique to obtain good (relaxed) candidate solutions, as described next.

\subsection{Joint optimization using Frank-Wolfe}
\label{subsec:frank-wolfe}
When dealing with a constrained optimization problem for which it is easy to solve linear programs but difficult to project on the feasible set, the Frank-Wolfe algorithm is an excellent choice~\cite{Jaggi2013,Lacoste15GlobalLinearFW}.
It is exactly the case for our relaxed problem, where the linear program over the convex hull of feasible integer matrices can be solved efficiently via dynamic
programming.
%
%
Moreover, \cite{lacoste16nonconvexFW} recently showed that the Frank-Wolfe algorithm with line-search converges to a stationary point for non-convex objectives at a rate of $O(1/\sqrt{k})$.
We thus use this algorithm for the joint optimization of~\eqref{eq:jointstateactionprob}. As the objective is quadratic, we can perform exact line-search analytically, which speeds up convergence in practice.
Finally, in order to get a good initialization for both variables $Z$ and $Y$, we first optimize separately $f(Z)$ and $g(Y)$ (without the non-convex $d(Z,Y)$), which are both convex functions.

\noindent\textbf{Dynamic program for the tracklets.}
In order to apply the Frank-Wolfe algorithm, we need to solve a linear program (LP) over our set of constraints. 
Previous work has explored ``\textit{exact one}" ordering constraints for time localization problems~\cite{Bojanowski14weakly}.
%
Differently here, we have to deal with the spatial (non overlap) constraint  \emph{and} finding ``\textit{at least one}" candidate tracklet per state.
To deal with these two requirements, we propose a novel dynamic programming approach.
First, the ``at least one" constraint is encoded by having a memory variable which indicates whether state~1 or state~2 have already been visited. This variable is used to propose valid state decisions for consecutive tracklets.
Second, the challenging ``non-overlap" tracklet constraint is included by constructing valid left-to-right paths in a cost matrix while carefully considering the possible authorized transitions.  
%
We provide details of the formulation in Appendix~\ref{app:dp}.
In addition, we show in section~\ref{sec:exp_res} that these new constraints are \emph{key} for the success of the method. %

%
%
%
%
%


%
%
%

%
%
%
%
%
%
%


\subsection{Joint rounding method}
\label{sec:rounding}
Once we obtain a candidate solution of the relaxed problem, we have to round it to an integer solution in order to make predictions.
Previous works~\cite{Alayrac15Unsupervised,Bojanowski15weakly} have observed that using the learned $W^*$ classifier for rounding gave better results than other possible alternatives. We extend this approach to our joint setup by proposing the following new rounding procedure. 
We optimize problem~\eqref{eq:jointstateactionprob} but fix the values of $W$ in the discriminative clustering costs. 
Specifically, we minimize the following cost function over the integer points $Z\in \mathcal{Z}$ and $Y \in \mathcal{Y}$:
\begin{align}
\label{eq:jcrrounding}
\!\!\!\!\!\frac{1}{2T} \|Z - X_{v} W^*_{v}\|_F^2 +  \frac{1}{2M} \|Y - X_{s} W^*_{s}\|_F^2 +  d(Z, Y), \!\!
\end{align}
where $W^*_{v}$ and $W^*_{s}$ are the classifier weights obtained at the end of the relaxed optimization.
Because $y^2 = y$ when $y$ is binary, \eqref{eq:jcrrounding} is actually a \emph{linear} objective over the binary matrix~$Y_n$ for~$Z_n$ fixed.
Thus we can optimize~\eqref{eq:jcrrounding} \emph{exactly} by solving a dynamic program on $Y_n$ for each of the $T_n$ possibilities of $Z_n$, yielding $O(M_n T_n)$ time complexity per clip (see Appendix~\ref{app:jcr} for details).



